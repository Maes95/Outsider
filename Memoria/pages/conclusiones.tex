La finalidad de este proyecto es, tanto aprender y hacer uso de tecnologías
asíncronas modernas, como la creación de una aplicación totalmente funcional 
que implemente estas tecnologías en detalle.

El objetivo principal es la creación de una aplicación que emule de la forma más
completa el juego de adivinanzas de palabras y roles que se propone. Además de ir implementando
diferentes reglas y funcionalidades al juego base, se proponen otros objetivos secundarios
como la implementación de testing o el despliegue de la aplicación a través de un proveedor de 
servicios. Se quiere profundizar en el uso de websockets y lo que implica su desarrollo, no
simplemente quedarse en la superficie, por ello, todo el desarrollo gira entorno a este eje
central. 

La aplicación final incluye múltiples sistemas interconectados que permiten su correcto
funcionamiento y uso por parte del usuario final a través de un navegador web. Para lograr esto, 
ha sido necesario trabajar con diversas tecnologías/frameworks e implementar la lógica y diseño 
necesarios para crear: Un servidor backend encargado de manejar la lógica websocket y un frontend
responsable de facilitar una interfaz web con todas las funcionalidades pertinentes. 
También ha sido necesario gestionar la comunicación necesaria entre los sistemas así como administrar
los elementos adicionales necesarios.

Para el trabajo relacionado con el backend se destaca el uso complejo y en detalle de tecnologías
websocket a través de Django Channels. Todos los mensajes de juego y la comunicación en tiempo real
entre jugadores no es una implementación sencilla y hay que tener en cuenta muchos 
aspectos de la comunicación asíncrona entre varios usuarios. Además de esta implementación,
se ha realizado trabajo adicional de testing, tanto para aprender el correcto funcionamiento 
de la pruebas automáticas en estos entornos, como para poder probar de forma más 
elaborada la aplicación.

Por otra parte, para el desarrollo de la interfaz web, se han implementado interfaces
usables mediante el uso de librerías de componentes (Vue y Vuetify) que facilitan mucho el trabajo de desarrollo
a la vez que ofrecen un resultado bastante atractivo. También se destaca toda la lógica de
intercomunicación y el código necesario implementado directamente en el frontend.

Además de la implementación, se destaca la preparación de la aplicación mediante Docker, 
para mejorar su portabilidad, y el despliegue final a través de AWS, 
lo que permite su fácil acceso y uso por parte de cualquier usuario.

Teniendo en cuenta lo explicado en los anteriores párrafos, se puede concluir
que el proyecto cumple los objetivos propuestos a lo largo del desarrollo.